\chapter{Unity-Package-Template: Automated Unity Development Infrastructure}

The inherent conflict between professional software engineering standards and the unity editor's project-centric nature necessitated the purposeful development of this production-ready architecture. the environment's dependability is crucial in a research-focused setting such as the \ac{CAS} simulator; if the underlying framework is non-deterministic or prone to human configuration errors, the scientific validity of the data produced is compromized. the emphasis was moved from handling unity's peculiar project structures to a modular, asset-oriented approach by designing a `template-first' methodology. by treating the simulator as a high-integrity library rather than a single executable, this methodology makes sure that the core functionality is independent of the particular unity version or project parameters being used for visualization.

\section{Requirements}

Requirements for the Unity-Package-Template are split into separated categories to ensure the infrastructure supports both the template developer and the end-user.

\subsection{Business Requirements}

\begin{itemize}
  \item The project shall provide a `Zero-Friction' onboarding process, enabling a developer to move from cloning to a functional development environment in a single step.
  \item The system must inject unique namespaces, names, and domains into the boilerplate code during initialization.
  \item The infrastructure shall automate repetitive tasks like Unity License activation and GPG key generation.
  \item The project must maintain a strict `Package-First' orientation, treating the code as a distributable library rather than a standalone executable.
  \begin{itemize}
    \item It must distinguish between Runtime and Editor content.
    \item It must make no assumptions about the final playable build or entry point.
  \end{itemize}
  \item The template must be self-evolving, allowing the infrastructure to be updated without breaking the projects that were instantiated from it.
\end{itemize}

\subsection{Domain Requirements}

\subsubsection{Unity Ecosystem Domain}

\begin{itemize}
  \item The system shall enable package development with the ease of the Unity Editor.
  \item The repository structure must adhere to the \ac{UPM} conventions \cite{unity_manual_package_types_2026} while keeping the distributable source separated from development `sandbox' artifacts.
  \item The system must handle the specific challenges of Unity asset version control, such as the merging of YAML-based files (scenes, prefabs, and metas).
  \item The environment must support the specific technical stack of Unity 6000 and .NET 8.
\end{itemize}

\subsubsection{DevOps Domain}

\begin{itemize}
  \item The infrastructure must guarantee `Deterministic Versioning', where version increments are tied to the intent of the changes.
  \item The repository must enforce no development secrets, local licenses, or temporary editor files are leaked into the public source.
  \item The system must enable `Policy as Code', making branching rules, quality gates, and release logic explicit and reviewable.
\end{itemize}

\subsubsection{Trust Domain}

\begin{itemize}
  \item The system must provide authenticity ensuring that every release is signed and can be audited back to a specific commit.
  \item The infrastructure must support `Living Documentation', where \ac{API} references and usage guides stay synchronized with the evolving code.
\end{itemize}

\subsection{Functional Requirements}

\begin{itemize}
  \item The system should allow a user to initialize a fully configured repository with a single command.
  \item The setup process must automatically configure project metadata (Domain, Company, Package Name) across all internal configurations.
  \item The system should handle the automated injection of required secrets for CI/CD (Unity licenses, analysis tokens) during the initial bootstrap.
  \item The environment should automatically configure the local version control system to handle Unity-specific merge logic.
  \item The system must provide automated validation for code style and static analysis.
  \item The infrastructure should automatically execute tests across different categories (Runtime and Editor) in a headless environment.
  \item The system must prevent the integration of code that does not meet the defined quality gates or commit message standards.
\end{itemize}

\subsection{Non-Functional Requirements}

\begin{itemize}
\item The infrastructure must be executable on any Linux-based environment or modern terminal emulator.
\item Setup failure must be `loud' and diagnostic, providing clear error messages when prerequisites are missing.
\end{itemize}

\section{Features}
