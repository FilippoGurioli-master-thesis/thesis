\chapter{Introduction}
%
The landscape of modern computing is undergoing a fundamental shift.
We are moving away from an era defined by powerful, isolated machines toward one characterized by massive, interconnected ensembles of devices. This transition is visible in everything from global \ac{IoT} sensor networks to smart city infrastructures. In these scenarios, the challenge is no longer just `how to compute', but rather `how to coordinate'.

As the number of devices in these systems grows into the thousands or millions, traditional centralized management becomes a bottleneck. The latency, bandwidth constraints, and single-point-of-failure risks of a `command-and-control' architecture make it unsuitable for the dynamic, often unpredictable environments these systems inhabit. Instead, we must look toward decentralized coordination, where collective intelligence arises from local interactions rather than global oversight.

This thesis explores the intersection of high-level collective programming and high-fidelity simulation. Specifically, it addresses the engineering gap between abstract coordination models, such as \ac{AC}, and the practical requirements of developing, testing, and deploying these models within realistic 3D environments. By leveraging the power of modern game engines and automated development workflows, this work aims to provide a robust infrastructure for the next generation of collective system design.

\section{Motivation: Swarm Behaviour}
\section{Problem Statement: Engineering Challanges in Simulation}
\section{Contributions}
\subsection{A Standardized Automation Workflow for Unity Packages}
\subsection{A 3D Aggregate Computing Simulator}
