\chapter{Collektive⨯Unity: Designing a 3D Simulator for Collective Systems}

This chapter face the core research project produced for this thesis: a simulator for 3D \ac{CAS}.

\section{Goal}

The project goal is to bridge the Unity game engine with the aggregate computing library named Collektive.

This communication should be bidirectional, achieve high performance and enable huge customization.

\section{Requirements}

Requirements are splitted into separated categories.

\subsection{Business Requirements}

\begin{itemize}
  \item The project should create a communication channel between the Collektive back-end and the Unity front-end.
  \item The communication should be bidirectional, i.e. there should be a way in which Unity communicates to Collektive information grasped from the environment and there should be a way in which Collektive answers to that communication.
  \item Between all the available implementations, the most performance compliant should be used.
  \item The integration should allow Collektive nodes to perceive Unity's colliders, rigidbodies and spatial triggers as first-class citizens.
  \item The integration layer should reamin agnostic to the specific \ac{CAS} case study.
\end{itemize}

\subsection{Domain Requirements}

\subsubsection{Simulator Requirements}

\begin{itemize}
  \item The simulator should have customizable node sensors
  \item The simulator should have customizable node actuators
  \item The simulator should have customizable step duration (i.e. \textit{delta time})
  \item The simulator should be pausable
  \item The simulator should have a centered handling of randomization to enable reproducibility
  \item The simulator should support addition and remotion of nodes in the simulation dynamically
  \item The simulator should allow nodes to interacts at least with the following unity components:
    \begin{itemize}
      \item rigid body
      \item collider
    \end{itemize}
\end{itemize}

\subsubsection{Communication Requirements}

\begin{itemize}
  \item The communication should follow the reactive pattern (i.e. Collektive reacts to Unity's stimuli).
  \item The data exchanged should be agnostic from the underlying case study.
  \item Performance should be the driver for choosing the right technology.
\end{itemize}

\subsection{Functional Requirements}

\subsubsection{User Functional Requirements}

\subsubsection{System Functional Requirements}

\subsection{Non-Functional Requirements}

\subsection{Implementation Requirements}

\section{Architecture}
