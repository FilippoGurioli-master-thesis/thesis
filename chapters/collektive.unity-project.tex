\chapter{Collektive⨯Unity: Designing a 3D Simulator for Collective Systems}

This chapter face the core research project produced for this thesis: a simulator for 3D \ac{CAS}.

\section{Goal}

The project goal is to bridge the Unity game engine with the aggregate computing library named Collektive.

This communication should be bidirectional, achieve high performance and enable huge customization.

\section{Requirements}

Requirements are splitted into separated categories.

\subsection{Business Requirements}

\begin{itemize}
  \item The project should create a communication channel between the Collektive back-end and the Unity front-end.
  \item The project should grasp information from Unity node sensors and share them to the Collektive program.
  \item The integration must map the output of the Collektive aggregate program to Unity Actuators (e.g., changing position, color, or state of a GameObject).
  \item The system shall evaluate and implement a low-latency \ac{IPC} or bridge mechanism to minimize overhead between the JVM-based Collektive and the C\#-based Unity environment.
  \item The integration should allow Collektive nodes to perceive Unity's colliders, rigidbodies and spatial triggers as first-class citizens.
  \item The integration layer should reamin agnostic to the specific \ac{CAS} case study.
\end{itemize}

\subsection{Domain Requirements}

\subsubsection{Simulator Domain}

\begin{itemize}
  \item The simulator should have customizable node sensors.
  \item The simulator should have customizable node actuators.
  \item The simulator should have customizable step duration (i.e. \textit{delta time}).
  \item The simulator should be able to pause the simulation.
  \item The simulator should have a centered handling of randomization to enable reproducibility.
  \item The simulator should support addition and remotion of nodes in the simulation dynamically.
  \item The simulator should allow nodes to interacts at least with the following unity components:
    \begin{itemize}
      \item rigid body
      \item collider
    \end{itemize}
  \item The simulator should support addition and remotion of neighbors dynamically.
  \item The simulator should support any kind of nighborhood discovery.
\end{itemize}

\subsubsection{Communication Domain}

\begin{itemize}
  \item The communication should follow the reactive pattern (i.e. Collektive reacts to Unity's stimuli).
  \item The data exchanged should be agnostic from the underlying case study.
  \item Performance should be the driver for choosing the right technology.
\end{itemize}

\subsubsection{Research Domain}

\begin{itemize}
  \item The system should prove the feasability of integrating game engines within \ac{CAS} frameworks.
\end{itemize}

\subsection{Functional Requirements}

\subsubsection{User Functional Requirements}

\begin{itemize}
  \item The user should treat a Unity scene as the simulation environment.
  \item The user should treat the Unity Editor as the simulator.
  \item The user should be able to add and remove nodes from the environment.
  \item The user should be able to create neighborhood discovery logic.
  \item The user should be able to inject any kind of collektive program inside the simulation.
  \item The user should be able to attach many different sensors and actuators to the same node.
  \item The user should be able to configure each node independently from the others.
\end{itemize}

\subsubsection{System Functional Requirements}

\begin{itemize}
  \item The system should allow users to define custom sensors and acutators without modifying the core integration library.
  \item The system should allow users to define custom Collektive program without modifying the core integration library.
\end{itemize}

\subsection{Non-Functional Requirements}

\begin{itemize}
  \item The system should maintain stable frame rate ( > 30 FPS) with at least 500 active collektive nodes in a low budget laptop (ryzen 7 5700U, 16GB DDR4, integrated GPU).
  \item The system should be implemented with the fastest technology found during exploration.
\end{itemize}

\section{Architecture}
